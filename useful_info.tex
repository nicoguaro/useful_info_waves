\documentclass[12pt,letterpaper,landscape]{article}
\usepackage{amsmath}
\usepackage{amsfonts}
\usepackage{amssymb}
\usepackage{graphicx}
\usepackage[left=2cm,right=2cm,top=2cm,bottom=2cm]{geometry}

%%%%%%%%%%%%%%%%%%%%%%%%%%%%%% User specified LaTeX commands.

%%%%%%%%%%%%%%%%%%%%%%%%%%%%%% Title

\title{\textbf{Useful information}}
\author{Nicol\'as Guar\'in-Zapata}
\date{\today}


%%%%%%%%%%%%%%%%%%%%%%%%%%%%%% Document
\begin{document}
\maketitle

\section{Relations between elastic constants}

\begin{table}[h]
\centering %
\begin{tabular}{|c|c|c|c|c|c|c|c|c|c|c|}
\hline 
 & $(K,E)$  & $(K,\lambda)$  & $(K,G)$  & $(K,\nu)$  & $(E,G)$  & $(E,\nu)$  & $(\nu,G)$  & $(\nu,\lambda)$  & $(G,\lambda)$  & $(G,M)$ \\
\hline 
$K=$  & $K$  & $K$  & $K$  & $K$  & $\frac{EG}{3(3G-E)}$  & $\frac{E}{3(1-2\nu)}$  & $\lambda+\frac{2G}{3}$  & $\frac{\lambda(1+\nu)}{3(1-2\nu)}$  & $\frac{2G(1+\nu)}{3(1-2\nu)}$  & $M-\frac{4G}{3}$ \\
\hline 
$E=$  & $E$  & $\frac{9K(K-\lambda)}{3K-\lambda}$  & $\frac{9KG}{2K+G}$  & $3K(1-2\nu)$  & $E$  & $E$  & $\frac{G(3\lambda+2G)}{\lambda+G}$  & $\frac{\lambda(1+\nu)(1-2\nu)}{\nu}$  & $2G(1+\nu)$  & $\frac{G(3-M-4G)}{M-2G}$ \\
\hline 
$\lambda=$  & $\frac{3K(3KE)}{9K-E}$  & $\lambda$  & $K-\frac{2G}{3}$  & $\frac{3K\nu}{1+\nu}$  & $\frac{G(E-2G)}{EG-E}$  & $\frac{E\nu}{(1+\nu)(1-2\nu)}$  & $\lambda$  & $\lambda$  & $\frac{2G\nu}{1-2\nu}$  & $M-2G$ \\
\hline 
$G=$  & $\frac{3KE}{9K-E}$  & $\frac{3(K-\lambda)}{2}$  & $G$  & $\frac{3K(1-2\nu)}{2(1+\nu)}$  & $G$  & $\frac{E}{2(1+\nu)}$  & $G$  & $\frac{\lambda(1-2\nu)}{2\nu}$  & $G$  & $G$ \\
\hline 
$\nu=$  & $\frac{3K-E}{6K}$  & $\frac{\lambda}{3K-\lambda}$  & $\frac{3K-2G}{2(3K+G)}$  & $\nu$  & $\frac{E}{2G}-1$  & $\nu$  & $\frac{\lambda}{2(\lambda+G)}$  & $\nu$  & $\nu$  & $\frac{M-2G}{2(M-G)}$ \\
\hline 
$M=$  & $\frac{3K(3K+E)}{9K-E}$  & $3K-2\lambda$  & $K+\frac{4G}{3}$  & $\frac{3K(1-\nu)}{1+\nu}$  & $\frac{G(4G-E)}{3G-E}$  & $\frac{E(1-\nu)}{(1+\nu)(1-2\nu)}$  & $\lambda+2G$  & $\frac{\lambda(1-\nu)}{\nu}$  & $\frac{2G(1-\nu)}{1-2\nu}$  & $M$ \\
\hline 
\end{tabular}
\end{table}


$K$: Bulk modulus, $\lambda$: Lam\'e's first parameter, $E$: Young's
modulus, $G$: Shear modulus, $\nu$: Poisson's ratio, $M$: P-wave
modulus.


\section{Relations for elastic wave speeds}

The P-wave is a dilatational wave with speed $\alpha$ given by 
\begin{align*}
 & \alpha^{2}=\frac{\lambda+2G}{\rho},\qquad\alpha^{2}=\frac{G(1-\nu)}{\rho},\\
 & \alpha^{2}=\frac{M}{\rho},\qquad\alpha^{2}=\frac{E(1-\nu)}{(1+\nu)(1-2\nu)\rho},\\
 & \alpha^{2}=\frac{2\beta^{2}(1-\nu)}{1-2\nu}.
\end{align*}
 The S-wave is a distorsional wave with speed $\beta$ given by 
\begin{align*}
 & \beta^{2}=\frac{G}{\rho}\\
 & \beta^{2}=\frac{E}{2(1+\nu)\rho},\\
 & \beta^{2}=\frac{\alpha^{2}(1-2\nu)}{2(1-\nu)}.
\end{align*}
 Some particular values for the ratio 
\[
\frac{\alpha^{2}}{\beta^{2}}=\frac{2(1-\nu)}{1-2\nu}
\]
 are 
\begin{align*}
 & \frac{\alpha^{2}}{\beta^{2}}=\frac{4}{3}\quad\mbox{for }\nu=-1\enspace,\\
 & \frac{\alpha^{2}}{\beta^{2}}=2\quad\mbox{for }\nu=0\enspace,\\
 & \frac{\alpha^{2}}{\beta^{2}}=4\quad\mbox{for }\nu=\frac{1}{3}\enspace,\\
 & \frac{\alpha^{2}}{\beta^{2}}\rightarrow\infty\quad\mbox{when }\nu\rightarrow\frac{1}{2}\enspace.
\end{align*}



\section{Courant-Friedrichs-Lewy condition}

The Courant-Friedrichs-Lewy condition (CFL condition) is a necessary
condition for convergence while solving certain partial differential
equations (usually hyperbolic PDEs) numerically by the method of finite
differences. It arises when explicit time-marching schemes are used
for the numerical solution. The condition is named after Richard Courant,
Kurt Friedrichs, and Hans Lewy who described it in their 1928 paper
\cite{CFL}.

The criterion could be stated as 
\begin{align*}
C=v_{x}\frac{\Delta t}{\Delta x}\leq C_{max}\qquad\mbox{in 1D}\enspace;\\
C=v_{x}\frac{\Delta t}{\Delta x}+v_{y}\frac{\Delta t}{\Delta y}\leq C_{max}\qquad\mbox{in 2D}\enspace;\\
C=v_{x}\frac{\Delta t}{\Delta x}+v_{y}\frac{\Delta t}{\Delta y}+v_{z}\frac{\Delta t}{\Delta z}\leq C_{max}\qquad\mbox{in 3D}\enspace;
\end{align*}
 Where $v_{x_{i}}$ is the phase speed (for wave phenomena) in the
$x_{i}$ direction, $\Delta x_{i}$ is the minimum spatial discretization
in $x_{i}$ direction, $\Delta t$ is the time step and $C_{max}$
is the maximum allowable value for $C$, which depends on the time
discretization scheme but should be less than 1.

In the case of elastodynamics and for a spatial discretization with
the finite element method, the criterion could be re-stated as 
\begin{align*}
C\leq\alpha\frac{\Delta t}{h}\leq C_{max}\qquad\mbox{in 1D}\enspace;\\
C\leq2\alpha\frac{\Delta t}{h}\leq C_{max}\qquad\mbox{in 2D}\enspace;\\
C\leq3\alpha\frac{\Delta t}{h}\leq C_{max}\qquad\mbox{in 3D}\enspace;
\end{align*}
 where $\alpha$ is the phase speed for the P-wave and $h$ is the
minimum distance between consecutive nodes. This give us the maximum
allowable timestep as 
\begin{align}
\Delta t\leq C_{max}\frac{h}{\alpha}\qquad\mbox{in 1D}\enspace;\\
\Delta t\leq\frac{C_{max}}{2}\frac{h}{\alpha}\qquad\mbox{in 2D}\enspace;\\
\Delta t\leq\frac{C_{max}}{3}\frac{h}{\alpha}\qquad\mbox{in 3D}\enspace.
\end{align}



\section{Nyquist criterion}

The Nyquist–Shannon sampling theorem, after Harry Nyquist and Claude
Shannon, in the literature more commonly referred to as the Nyquist
sampling theorem or simply as the sampling theorem, is a fundamental
result in the field of information theory, in particular telecommunications
and signal processing. Sampling is the process of converting a signal
(for example, a function of continuous time or space) into a numeric
sequence (a function of discrete time or space). Shannon's version
of the theorem states \cite{Shannon}:
\begin{quote}
If a function $x(t)$ contains no frequencies higher than $B$ hertz,
it is completely determined by giving its ordinates at a series of
points spaced $1/(2B)$ seconds apart. 
\end{quote}
This theorem implies for us in the numerical simulation of wave propagation
that 
\[
h\leq\frac{\lambda}{2}\enspace,
\]
 where $h$ is the maximum distance between consecutive nodes and
$\lambda$ is the shortest wavelength that want to be sampled. So,
the selection of $h$ is commonly 
\[
h=\frac{\lambda}{k}\enspace,
\]
 where $k>2$ is a factor that depends on the numerical method. For
finite element methods $k$ is commonly 10, and for spectral element
methods $k$ is reduced to 5 \cite{Komatitsch99}.


\section{Ricker wavelet}

It is common to use a Ricker wavelet as the input signal in simulations
since it has the energy concentrated around a circular frequency $\omega_{c}$.
Let's write the signal as 
\[
f(t)=\left(\frac{12t^{2}}{b^{2}}-1\right)e^{-\frac{6t^{2}}{b^{2}}}\enspace,
\]
 where $b$ is the elapsed time between the peaks in the time domain
\cite{Papageorgiou91}. The Fourier transform for this signal is 
\[
\hat{f}(\omega)=-\frac{\sqrt{\pi}b^{3}\omega^{2}e^{-\frac{b^{2}\omega^{2}}{24}}}{2\ 6^{3/2}}\enspace,
\]
 with a characteristic (maximum) circular frequency 
\[
\omega_{c}=\frac{2\sqrt{6}}{b}.
\]
 To satisfy the sampling theorem we need to design our simulations
thinking about the shortest wavelength, so we need to think about
the maximum frequency. To know where to trunk the signal (in the frequency
domain) we could compute how much energy is in $[0,\omega_{max}]$
and compare this value with the energy stored in the interval $[0,\infty)$.
So, the ratio 
\[
R(\omega_{max})=\frac{\int\limits _{0}^{\omega_{max}}\hat{f}(\omega)d\omega}{\int\limits _{0}^{\infty}\hat{f}(\omega)d\omega}\times100\%\enspace,
\]
 tell us the proportion of energy taken into account if we neglect
circular frequencies higher than $\omega_{max}$. Some particular
values are : 
\begin{align}
 & R(1.0\omega_{c})=42.76\%\enspace;\\
 & R(1.5\omega_{c})=78.77\%\enspace;\\
 & R(2.0\omega_{c})=95.40\%\enspace;\\
 & R(3.0\omega_{c})=99.96\%\enspace;\\
 & R(4.0\omega_{c})=99.99\%\enspace.
\end{align}

\begin{thebibliography}{1}
\bibitem{CFL} Courant, R.; Friedrichs, K.; Lewy, H. (1928), \emph{\"Uber
die partiellen Differenzengleichungen der mathematischen Physik} (in
German), Mathematische Annalen 100 (1): 32--74.

\bibitem{Shannon} C. E. Shannon, \emph{Communication in the presence
of noise}, Proc. Institute of Radio Engineers, vol. 37, no. 1, pp.
10--21, Jan. 1949. Reprint as classic paper in: Proc. IEEE, vol. 86,
no. 2, (Feb. 1998).

\bibitem{Komatitsch99} D. Komatitsch and J. Tromp,\emph{Introduction
to the spectral element method for three-dimensional seismic wave
propagation}, Geophysical Journal International, 1999, (139): 806--822.

\bibitem{Papageorgiou91} A. Papageorgiou and J. Kim,\emph{Stufy of
the propagation and amplification of seismic waves in Caracas Valley
with reference to the 29 July 1967 earthquake: SH waves}, Bulletin
of the Seismological Society of America, 1991, 81 (6): 2214--2233. \end{thebibliography}

\end{document}